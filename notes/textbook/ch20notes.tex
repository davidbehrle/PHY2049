\documentclass[a4paper]{article}
\usepackage[english]{babel}
\usepackage[utf8]{inputenc}
\usepackage{textcomp}
\usepackage{amsmath}
\usepackage{amssymb}
\usepackage{gensymb}
\usepackage{physics}
\usepackage{graphicx}
\usepackage[colorinlistoftodos]{todonotes}
\usepackage{xcolor}
\usepackage{array}
\usepackage{tabularx}
\usepackage{tikz}
\usepackage{pgfplots}
\usepackage{framed}
\usepackage{xfrac}
\usepackage[most]{tcolorbox}
\usepackage{fix-cm}
\usepackage{cancel}
\usepackage{pagecolor}
\usepackage[margin=0.5in]{geometry}
\usetikzlibrary{quotes,angles}
\usetikzlibrary{decorations.pathreplacing}
\usetikzlibrary{calc}
\usetikzlibrary{external}
\tikzexternalize[prefix=tikz/ch20figures/]
\usepgfplotslibrary{fillbetween}

\let\phi\varphi
\let\bf\textbf
\let\la\langle
\let\ra\rangle
\pgfplotsset{compat=1.18}
\newcommand\der[2]{\frac{d #1}{d #2}}
\newcommand\Deltat{\Delta t}
\newcommand\rads{\text{ rad\;s}^{-1}}
\newcommand\radss{\text{ rad\;s}^{-2}}
\newcommand\rad{\text{ rad}}
\newcommand\s{\text{ s}}
\newcommand\m{\text{ m}}
\newcommand\km{\text{ km}}
\newcommand\J{\text{ J}}
\newcommand\Nm{\text{ Nm}}
\newcommand\ms{\text{ ms}^{-1}}
\newcommand\mss{\text{ ms}^{-2}}
\newcommand\kg{\text{ kg}}
\newcommand\kgms{\text{ kg\;ms}^{-1}}
\newcommand\kgmm{\text{ kg\;m}^{2}}
\newcommand\kgmms{\text{ kg\;m}^2\text{s}^{-1}}
\newcommand{\ih}{\hat{\imath}}
\newcommand{\jh}{\hat{\jmath}}
\newcommand{\AxisRotator}[1][rotate=0]{%
    \tikz [x=0.25cm,y=0.60cm,line width=.2ex,-stealth,#1] \draw (0,0) arc (-150:150:1 and 1);%
}
\def\centerarc[#1](#2)(#3:#4:#5){\draw[#1] ($(#2)+({#5*cos(#3)},{#5*sin(#3)})$) arc (#3:#4:#5)}
% Syntax: \centerarc[draw options] (center) (initial angle:final angle:radius);

\title{Electric Charge, Force, \& Field}
\author{Essential University Physics Vol. 2, 4e.}
\date{}

\definecolor{fg0}{HTML}{fbf1c7}

\makeatletter % change only the display of \thepage, but not \thepage itself:
\patchcmd{\ps@plain}{\thepage}{\textcolor{fg1}{\thepage}}{}{}
\makeatother

\begin{document}
\pagestyle{plain}

\definecolor{bg0}{HTML}{282828}
\definecolor{bg0_h}{HTML}{1d2021}
\definecolor{bg0_s}{HTML}{32302f}
\definecolor{bg1}{HTML}{3c3836}
\definecolor{bg2}{HTML}{504945}
\definecolor{bg3}{HTML}{665c54}
\definecolor{bg4}{HTML}{7c6f64}
\definecolor{fg1}{HTML}{ebdbb2}
\definecolor{fg2}{HTML}{d5c4a1}
\definecolor{fg3}{HTML}{bdae93}
\definecolor{fg4}{HTML}{a89984}
\definecolor{gbaqua}{HTML}{8ec07c}
\definecolor{gbaqua2}{HTML}{689d6a}
\definecolor{gbred}{HTML}{fb4934}
\definecolor{gbred2}{HTML}{cc241d}
\definecolor{gbgreen}{HTML}{b8bb26}
\definecolor{gbgreen2}{HTML}{98971a}
\definecolor{gbyellow}{HTML}{fabd2f}
\definecolor{gbyellow2}{HTML}{d79921}
\definecolor{gbblue}{HTML}{83a598}
\definecolor{gbblue2}{HTML}{458588}
\definecolor{gbpurple}{HTML}{d3869b}
\definecolor{gbpurple2}{HTML}{b16286}
\definecolor{gborange}{HTML}{fe8019}
\definecolor{gborange2}{HTML}{d65d0e}
\definecolor{gbgray}{HTML}{a89984}
\definecolor{gbgray2}{HTML}{928374}
\color{fg0}
\pagecolor{bg0_s}
\colorlet{shadecolor}{gbaqua2}

\setcounter{section}{20}
\maketitle

\subsection{Electric Charge}
Electric charge is an intrinsic property of protons and electrons and comes in two varieties, \textit{positive} and \textit{negative}. These names are useful as the total charge (net charge) of an object is the algebraic sum of its constituent charges. Like charges repel, opposite charges attract, this constitutes a qualitative description of the electric force.
\vspace{1mm}\\
\bf{Quantities of Charge}
\vspace{1mm}\\
All electrons carry the same charge, as do all protons. The charge of a proton is \textit{exactly} the same magnitude as the electron's but with opposite sign. The magnitude of the electron or proton charge is the elementary charge $e$, and is quantized. Elementary particle theories show that the fundamental charge is actually $\frac{1}{3}e$ and resides on quarks, the building blocks of protons and neutrons among other particles. Quarks always join together to form particles with integer multiples of the full elementary charge $e$, and it seems impossible to isolate individual quarks.
\vspace{1mm}\\
The SI unit of charge is the coulomb (C), named after the French physicist Charles Augustin de Coulomb. From the late 19\textsuperscript{th} to early 21\textsuperscript{st} century, the coulomb was defined in terms of electric current and time, a definition that was difficult to implement. The 2019 revision of the SI defined the coulomb more simply by defining the elementary charge as $e = 1.602176634 \times 10^{-19}$ C. The coulomb is therefore the number of elementary charges equal to the inverse of this number: C $\approx 6.24 \times 10^{19}\ e$.
\vspace{1mm}\\
\bf{Charge Conservation}
\vspace{1mm}\\
Electric charge is a conserved quantity, the net charge in a closed region remains constant. Charged particles may be created or annihilated, but always in pairs of equal and opposite charge. The net charge is always the same.
\subsection{Coulomb's Law}
Attraction and repulsion of electric charges implies a force. Joseph Priestley and Charles Augustin de Coulomb investigated this force in the late 1700s and found that the force between two charges acts along the line joining them, with the magnitude proportional to the product of the charges and inversely proportional to the square of the distance between them. Coulomb's law summarizes these results:
\begin{equation}
    \vec{F}_{12} = \frac{kq_1q_2}{r^2}\hat{r}
\end{equation}
where $\vec{F}_{12}$ is the force charge $q_1$ exerts on $q_2$ and $r$ is the distance between the charges. In SI, the proportionality constant $k$ has the approximate value $9 \times 10^9\; \text{N}\; \text{m}^2\; \text{C}^{-2}$. Force is a vector, and $\hat{r}$ is a unit vector that helps determine its direction, pointing \textit{from} $q_1$ \textit{toward} $q_2$. Reversing the roles of $q_1$ and $q_2$, $\vec{F}_{21}$ has the same magnitude as $\vec{F}_{12}$ but the opposite direction, thus Coulomb's law obeys Newton's third law. The force is in the same direction as the unit vector when the charges have the same sign, and opposite the unit vector when the charges have opposite signs, accounting for the fact that like charges repel and opposite charges attract.
\begin{center}
    \begin{tikzpicture}
        \draw[->,very thick,-latex,gbblue2] (1,0.9)--node[below,color=fg1]{$\vec{F}_{12}$} (3,0.9);
        \draw[->,very thick,-latex,gbaqua2] (1,1.1)--node[above,color=fg1]{$\hat{r}$}(2,1.1);
        \filldraw[gbblue] (-1,1) circle (0.15);
        \filldraw[gbblue] (1,1) circle (0.15);
        \draw[thin] (-0.9,1)--(-1.1,1);
        \draw[thin] (-1,1.1)--(-1,0.9);
        \draw[thin] (0.9,1)--(1.1,1);
        \draw[thin] (1,1.1)--(1,0.9);
        \draw[thick] (-1,1) circle (0.15);
        \draw[thick] (1,1) circle (0.15);
        \draw[thick] (-1,0.75)--(-1,0);
        \draw[thick] (1,0.75)--(1,0);
        \draw[thick] (-1,0.1)--node[fill=bg0_s]{$r$} (1,0.1);
        \node at (0,-0.5) {(a)};
    \end{tikzpicture}
    \hspace{15mm}
    \begin{tikzpicture}
        \draw[->,very thick,-latex,gbblue2] (1,1)--node[above,color=fg1]{$\vec{F}_{12}$} (-0.85,1);
        \draw[->,very thick,-latex,gbaqua2] (1,1)--node[above,color=fg1]{$\hat{r}$}(2,1);
        \filldraw[gbblue] (-1,1) circle (0.15);
        \filldraw[gbred] (1,1) circle (0.15);
        \draw[thin] (-0.9,1)--(-1.1,1);
        \draw[thin] (-1,1.1)--(-1,0.9);
        \draw[thin] (0.9,1)--(1.1,1);
        \draw[thick] (-1,1) circle (0.15);
        \draw[thick] (1,1) circle (0.15);
        \draw[thick] (-1,0.75)--(-1,0);
        \draw[thick] (1,0.75)--(1,0);
        \draw[thick] (-1,0.1)--node[fill=bg0_s]{$r$} (1,0.1);
        \node at (0,-0.5) {(b)};
    \end{tikzpicture}
\end{center}
\bf{Point Charges \& the Superposition Principle}
\vspace{1mm}\\
Coulomb's law is only strictly true for point charges (charged objects of negligible size) which electrons and protons can be usually be treated as. Any two charged objects can also be considered as such if the distance between them is large compared to their size. Often the electric effects of charge distributions - arrangements of charge spread over space - are more interesting. To find the electric effect of such charge distributions, the effects of two or more charges must be combined.
\begin{center}
    \begin{tikzpicture}[scale=1.5]
        \draw[->,ultra thick,-latex,gbblue2] (0,0)--node[below,color=fg1]{$\vec{F}_{13}$}(1,0);
        \draw[->,ultra thick,-latex,gbblue2] (0,0)--node[right,color=fg1]{$\vec{F}_{\text{net}} = \vec{F}_{13} + \vec{F}_{23}$}({cos(60)},{sin(60)});
        \draw[->,ultra thick,-latex,gbblue2] (0,0)--node[left,color=fg1]{$\vec{F}_{23}$}({-cos(60)},{sin(60)});
        \filldraw[gbblue] (0,0) circle (0.1);
        \draw[thick] (0,0) circle (0.1);
        \node[below] at (0,-0.1){$q_3$};
        \draw[thin] (0,-0.075)--(0,0.075);
        \draw[thin] (-0.075,0)--(0.075,0);
        \filldraw[gbred] ({-1.5*cos(60)},{1.5*sin(60)}) circle (0.1);
        \draw[thin] ({-1.5*cos(60) - 0.075},{1.5*sin(60)})--({-1.5*cos(60) + 0.075},{1.5*sin(60)});
        \draw[thick] ({-1.5*cos(60)},{1.5*sin(60)}) circle (0.1);
        \node[below] at ({-1.5*cos(60)},{1.5*sin(60) - 0.1}){$q_2$};
        \filldraw[gbblue] (-1,0) circle (0.1);
        \draw[thin] (-1.075,0)--(-0.925,0);
        \draw[thin] (-1,0.075)--(-1,-0.075);
        \node[below] at (-1,-0.1){$q_1$};
        \draw[thick] (-1,0) circle (0.1);
    \end{tikzpicture}
\end{center}
The figure shows two charges $q_1$ and $q_2$ constituting a simple charge distribution. To calculate the net force these exert on a third charge $q_3$, calculate the forces $\vec{F}_{13}$ and $\vec{F}_{23}$ and add the two vectors. The force $q_1$ exerts on $q_3$ is unaffected by the presence pf $q_2$ and vice versa. Coulomb's law can be applied separately to the pairs $q_1q_3$ and $q_2q_3$ and the results combined. This fact - that electric forces add vectorially - is called the superposition principle.
\subsection{The Electric Field}
The gravitational field at a point is defined as the gravitational force per unit mass that an object at that point would experience. In this context, $\vec{g}$ can be thought of as the force per unit mass that any object would experience due to gravity. The gravitational field can be pictured as a continuous set of vectors that give the magnitude and direction of the gravitational force per unit mass at each point. The same can be done with the electric force, defining the electric field as the force per unit charge:
\begin{equation}
    \vec{E} = \frac{\vec{F}}{q}
\end{equation}
This equation can be used as a prescription for measuring electric fields. Place a point charge at some location, measure the electric force it experiencesm and divide by the change to get the field.
\begin{center}
    \begin{tikzpicture}[scale=1.5]
        \draw[thick] (0,0)--(2,0);
        \node[right] at (2,0.8){$\vec{g}$};
        \node at (0,-1.125){};
        \foreach \i in {0.1,0.3,...,1.9}{
            \draw[->,very thick,-latex,gbblue2] (\i,0.75)--(\i,0.1);
            \draw[->,very thick,-latex,gbblue2] (\i,1.5)--(\i,0.85);
            \filldraw (\i,0.75) circle (0.05);
            \filldraw (\i,1.5) circle (0.05);
        }
    \end{tikzpicture}
    \hspace{15mm}
    \begin{tikzpicture}[scale=1.5]
        \filldraw[gbblue] (0,0) circle (0.15);
        \draw[thick] (0,0) circle (0.15);
        \draw[thin] (-0.1,0)--(0.1,0);
        \draw[thin] (0,-0.1)--(0,0.1);
        \foreach \i in {22.5,67.5,112.5,...,337.5}{
            \draw[->,very thick,-latex] ({0.5*cos(\i)},{0.5*sin(\i)})--({1.5*cos(\i)},{1.5*sin(\i)});
            \filldraw ({0.5*cos(\i)},{0.5*sin(\i)}) circle (0.05);
            \draw[->,very thick,-latex] ({1.6*cos(\i)},{1.6*sin(\i)})--({2.25*cos(\i)},{2.25*sin(\i)});
            \filldraw ({1.6*cos(\i)},{1.6*sin(\i)}) circle (0.05);
        };
        \foreach \j in {0,45,90,...,315}{
            \draw[->,very thick,-latex] ({0.35*cos(\j)},{0.35*sin(\j)})--({2.25*cos(\j)},{2.25*sin(\j)});
            \filldraw ({0.35*cos(\j)},{0.35*sin(\j)}) circle (0.05);
        }
        \node at ({1.2*cos(58)},{1.2*sin(58)}){$\vec{E}_1$};
        \node at ({1.9*cos(32)},{1.9*sin(32)}){$\vec{E}_2$};
    \end{tikzpicture}
\end{center}
At the point at the tail of $\vec{E}_1$, the electric field is described by the vector $\vec{E}_1$ meaning a point charge $q$ placed there would experience an electric force $q\vec{E}_1$. Farther from the charge at the tail of $\vec{E}_2$, a point charge $q$ would experience a weaker force $q\vec{E}_2$.
\vspace{1mm}\\
If the electric field $\vec{E}$ at a point is known, equation 2 can be rearranged to find the force on any point charge $q$ placed at that point:
\begin{equation*}
    \vec{F} = q\vec{E}
\end{equation*}
If the charge $q$ is positive, the force is in the same direction as the field, if $q$ is negative, then the force is opposite to the direction of the field. The units of electric field are newtons per coulomb (N$\;$C$^{-1}$). Fields of hundreds or thousands of N$\;$C$^{-1}$ are common, while fields of 3 MN$\;$C$^{-1}$ will tear electrons from air molecules.
\vspace{1mm}\\
\bf{The Field of a Point Charge}
\vspace{1mm}\\
Once the field of a charge distribution is known, its effect on other charges can be calculated. Coulomb's law gives the force on a test charge $q_{\text{test}}$ located a distance $r$ from a point charge $q$: $\vec{F} = (\sfrac{kqq_{\text{test}}}{r^2})\hat{r}$, where $\hat{r}$ is a unit vector pointing away from $q$. The electric field arising from $q$ is the force per unit charge, or:
\begin{equation}
    \vec{E} = \frac{\vec{F}}{q_{\text{test}}} = \frac{kq}{r^2}\hat{r}
\end{equation}
Because of its similarity, this equation is also referred to as Coulomb's law. The equation contains no reference to the test charge $q_{\text{test}}$ because the field of $q$ exists independently of any other charge. Since $\hat{r}$ always points away from $q$, the direction of $\vec{E}$ is radially outward if $q$ is positive and radially inward if $q$ is negative.
\begin{center}
    \begin{tikzpicture}
        \filldraw[gbred] (0,0) circle (0.1);
        \draw[thick] (0,0) circle (0.1);
        \draw[thin] (-0.07,0)--(0.07,0);
        \foreach \i in {0,45,90,...,315}{
            \draw[->,very thick,-latex] ({cos(\i)},{sin(\i)})--({0.2*cos(\i)},{0.2*sin(\i)});
            \draw[->,very thick,-latex] ({1.75*cos(\i)},{1.75*sin(\i)})--({1.1*cos(\i)},{1.1*sin(\i)});
            \filldraw ({cos(\i)},{sin(\i)}) circle (0.05);
            \filldraw ({1.75*cos(\i)},{1.75*sin(\i)}) circle (0.05);
        }
        \foreach \j in {22.5,67.5,112.5,...,337.5}{
            \draw[->,very thick,-latex] ({1.6*cos(\j)},{1.6*sin(\j)})--({0.8*cos(\j)},{0.8*sin(\j)});
            \filldraw ({1.6*cos(\j)},{1.6*sin(\j)}) circle (0.05);
        }
    \end{tikzpicture}
\end{center}

\end{document}